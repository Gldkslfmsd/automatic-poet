\documentclass[a4]{article}
%\usepackage{czech}
\usepackage{graphicx}
\usepackage[utf8]{inputenc}   % pro unicode UTF-8
\usepackage{booktabs}
\usepackage{hyperref}
%#\usepackage{qtree}
%\usepackage{graphviz}
\usepackage{authblk}
%\usepackage{tikz-dependency}
\usepackage[ampersand]{easylist}

\def\furl#1{\footnote{\url{#1}}}


\begin{document}

\title{Automatic accentual-syllabic poet}

\author{Dominik Macháček}
\affil{
Saarland University, Saarbrücken, Germany,
dominik.machacek@matfyz.cz
}

\date{\today}

%\pacs{PACS numbers go here. These are classification codes for your  research. See {\tt http://publish.aps.org/PACS/} for more info.}
\maketitle

%\begin{abstract}
%TODO: An abstract is a great convenience for the reader and is required by all journals.
%\end{abstract}


\section{Introduction}

Our task is to create a generator of an accentual-syllabic poetry.

Verses in this kind of poetry are restricted by the rhythm of accents, all
verses must have given number of syllables and the verses in a strophe
rhyme by the chosen pattern. See example in figure \ref{may}.

\def\surl#1{%
    {\footnotesize\url{#1}}%
}%

\begin{figure}[ht]
\label{may}
\centerline{
I {\bf know} a {\bf glade}, spring {\bf crys}tal {\bf clear},
}
\centerline{
in {\bf dee}pest {\bf wood}land, {\bf crowned}
}
\centerline{
by {\bf sha}dy {\bf ferns} in {\bf si}lhou{\bf ettes},
}
\centerline{
red {\bf hea}ther {\bf all} a{\bf round}.\footnotemark
}
\caption{Bold syllables are stressed, other syllables are non-stressed.
There is a regular pattern of stressed and non-stressed
syllables in each verse. Second and fourth verse rhyme.
}
\end{figure}


\footnotetext{From "The Crystal Spring" by J. V. Sládek, translated by
Václav Z J Pinkava.
Available at \url{http://www.vzjp.cz/basne.htm\#Sladek}.
}


% Summary?


In this project we make an automatic generator of accentual-syllabic
poems. Its input are a poetic form, namely a pattern of stressed and
non-stressed syllables in each verse, and a raw coherent Czech text. The
whole process of generating is fully automatic, no manuall anotation of
the text is needed.

On the output there are poems compounded from newly-created nonsense words
reminding Czech language. They are pronouncable and they precisely
fit into the given poetic form.

\section{Motivation}

Accentual-syllabic poetry is very popular genre in many national cultures
including Czech, English and German. Poems written by human
poets are usually written in some variety of natural language, their words
have meaning and the whole poem has some intention.
To write a good poem is a difficult task, a poet must choose proper words
and fit them into a pleasant form. 

We want to make an automatic poems generator because with it new poems can
be easily created by computer. They will be unique, interesting and
entertaining pieces of art. They could also be published for general public
and draw attention to the whole field of Computational Linguistics.

\section{Related works}

We can find works from other authors more or less related to our topic, but
there isn't any work providing accentual-syllabic poetry generator for Czech. 
Related works are described in this section.

\subsection{Poem generators with restricted originality}

On the Internet we can find several
sites called "poems generators", which are in fact simple games producing texts with
very restricted originality. They ask a user for a set of input words and then
they put them into a static template. Results usually don't have a fixed rhythm
and they usually don't rhyme. See table \ref{tab:gen} for example of such generators.

\begin{table}[ht]
\begin{tabular}{ll}
\hline
{\bf name} & {\bf URL} \\
\hline
\hline
AI poem & \surl{http://www.aipoem.com/easypoem/} \\
Poem Generator & \surl{http://thinkzone.wlonk.com/PoemGen/PoemGen.htm} \\
PoemOfQuotes & \surl{http://www.poemofquotes.com/tools/poetry-generator.php} \\
\hline
\hline
\end{tabular}
\caption{}
\label{tab:gen}
\end{table}


\subsection{Short poems of restricted form}

There are restricted poem forms having small length and a simple constraints
verifiable by computer.
Computer programs can be used to generate or to seek them in a big corpus.
For example snowballs are sentences where every word is one letter longer
than previous. A computer can find such words in a corpus and put them
together to create a meaningful sentence. 
Summary\footnote{We have taken most of them from following source:
\url{http://mentalfloss.com/article/57715/14-hilarious-automatic-text-and-tweet-generators-flair-poetry-and-language-play}  
}
of such works is in table \ref{tab:restricted}.

\begin{table}[ht]
\begin{tabular}{lll}
\hline
{\bf name} & {\bf URL} & {\bf form} \\
\hline
\hline
Snowball poetry & \surl{https://twitter.com/snowballpoetry}
& snowball \\
Pentametron & \surl{https://twitter.com/pentametron} & iambic pentameter \\
Anagramatron & \surl{https://twitter.com/anagramatron} & anagrams \\
HAIKU9000 & \surl{https://twitter.com/HAIKU9000} & haiku \\
Pangramtweets & \surl{https://twitter.com/PangramTweets} & pangrams \\
\hline
\hline
\end{tabular}
\caption{}
\label{tab:restricted}
\end{table}

\subsection{Neural network poetry}

In other group of works state-of-the art deep learning techniques are used
to generate poetry. They put effort to make a poem with a deep meaning, but
use a loose form of a free verse. Some of them also aim to artificially
create poems which would pass Turing test, it means to be judged by humans
as created by humans and not by bot. They can be found on the site "Bot or
not"\furl{http://botpoet.com/}.

Some of the works related with neural network poetry can be found in
table\ref{tab:nn}.

\begin{table}[ht]
\begin{tabular}{lll}
\hline
{\bf name} & {\bf URL} \\
\hline
\hline
A Neural Network's Poetry  & \surl{http://neuralnetpoetry.blogspot.de/} \\
NeuralSnap & \surl{https://github.com/rossgoodwin/neuralsnap} \\
\hline
\hline
\end{tabular}
\caption{}
\label{tab:restricted}
\end{table}

% Slovak generator of Vogon poetry: http://www.ludoslovensky.sk/slova/vogon/
% could be also somehow mentioned


\subsection{Poetweet}

There exists a site poetweet\furl{http://poetweet.com.br/}, where a user
fills a Twitter channel and then chooses a poem type. He can choose either a sonnet,
rondel or indriso, that are a fixed forms of accentual-syllabic poetry.
Afterwards, tweets from the Twitter channel are extracted and a poem is
created from tweets' coherent excerpts. It also provides links for
mentioned tweets.

Poetweet doesn't have any restrictions on the channel's language and
doesn't provide any information about its inner design. It could be
intended for Portuegese and could use language-specific processes. Also the
poem form and source domain are very restricted.

Poems from Czech Twitter channels doesn't have a strict good-quality rhythm,
but they have rhymes.
%Poetweet also shows that using coherent text excerpts
%is a good idea, .


\section{Solution}

Generator's workflow consists of two phases, training and generating. 

In
training phase, an input text is preprocessed and splitted to syllables.
Then stresses are indicated.  Finally, a language model is created from list
of syllables and their accents.

In generating phase, a language model is used for creating of a poem by given form.

More detailed descriptions of each mentioned process follow in special subsections.

Our generator can work for arbitrary natural language, but we work with
Czech. Reasons for this decision is that there exist many
resources of Czech texts, we can use an automatic tool for splitting text to syllables,
and the word-stress in Czech language is very regular so we can indicate
it automatically. Czech is also a native language of the author of this
project, so we are able to evaluate the quality of resulting poems. 

For some other languages we should use either a special tools for splitting
to syllables and for accent indication (which we don't have for any other
language including English),
or we could use a corpus annotated with syllables and accents, but this
restricts the domain and size of input texts.


\subsection{Text preprocessing}

In text preprocessing we do tokenization and sentence segmentation, because
following syllabification and accentification steps require single sentences. We also
remove punctuation and digits and transform text to lowercase. This decreases number of
unique syllables and makes language model more inovative, although it losts
some information.

For tokenization and segmentation we use NLTK Punkt tokenizer \cite{nltk}. % reference

\subsection{Syllabification}

For automatic syllabification we used Sekáček\cite{sekacek}.
It's a Python implementation of static rules created by experts on this
issue. The algorithm was described by Jitka Štindlová\cite{naserec}.
It's intended only for Czech.

\subsection{Accents}

Resource as\cite{prizvuk} state that accents Czech is a language with
a fixed accent. It's always on the first syllable of a word some exceptions: 

-- monosylabic prepositions are so called enclitics, they take accent from
following word and therefore they're accented. 

%Example without
%prepostition: {\bf o}ke{\sl nní} {\bf
%rám}, with prepostion {\bf pod} o{\sl ke}nním {\bf rá}mem (a window frame, under a window
%frame)

-- some functions words as monosylabic (short forms of) personal and
reflexive pronouns or past tense and conditional's auxiliaries are clitics,
they always follow some word and are not accented %Example: 

-- in Czech poetry, a secondary accent is sometimes used, it's on the third and
fifth and every following odd syllable of a word

-- according to Rýmy.cz\cite{rymy}, monosyllabic words in poetry can be
sometimes read as accented and sometimes as non-accented, it depends on
it's position between other words

For our application we decided to implement only this rules. They hold in
majority of situations, but not always. For example, Jiří Zeman\cite{zeman} claims that
there are also articulatory influences leading to unstressed
pronunciation of monosylabic prepositions {\sl u, o} (near, about), if
following word starts with a vowel. %Compound words have also a secondary stress on
%a first syllable of second word, even if it

\subsubsection{Implementation}

We implemented a finite state transducer\cite{jurafsky} to mark every syllable as either 

-- primary stressed

-- unstressed

-- secondary stressed -- it's every third or following odd syllable in
a word and also monosylabic word apart clitics

In a stress pattern, secondary stressed syllables can hold as either
stressed or unstressed.  
For clitics detection we created a static list.


\subsection{Language model}

We used N-gram language model. One unit of model is
a syllable and its accent. Syllables have 

Example of model is in figure\ref{fig:model}.

\begin{figure}[ht]
\centerline{
Input text:
"nanana na nana"
Syllables:
%" na/P", "na/U", "na/S", " na /S", " na/P", "na /U"
}
\caption{
}
\label{fig:model}
\end{figure}


\footnotetext{From "The Crystal Spring" by J. V. Sládek, translated by
Václav Z J Pinkava.
Available at \url{http://www.vzjp.cz/basne.htm\#Sladek}.
}




\subsection{Generating}

When model is created

\subsubsection{Rhymes}


Rýmy.cz\cite{rymy} describes good-quality criteria of rhymes. Rhyme
shouldn't be trivial, for example a rhyme of two words of the same
grammatical category in the same grammatical form is considered as banal
and ugly. In our application, words have no meanings and therefore no
grammatical category.

Two words rhyme if the phonetical forms of their endings are similar. 
Therefore for a good rhyming we need a morphological tagging and
phonological transcription. We decided to avoid this issue by simple way:
If two verses should rhyme, then we simply replace the last syllable of the
second verse with the last syllable of the first verse.

\section{Example outputs}

\section{Conclusion}


\begin{thebibliography}{99}

\bibitem{nltk}
 {\sl NLTK: Nature Language Toolkit}

\bibitem{sekacek}
 {\sl Sekáček -- split Czech text to syllables}.
 \url{https://github.com/Gldkslfmsd/sekacek}


\bibitem{naserec}
 {\sl Jitka Štindlová: Dělení slov v češtině pomocí strojů}
(Word-splitting in Czech by machines). [1968] Published in {\sl Naše řeč}
(Our language),
{\sl year 51, number 1, pg. 23-32}. Available at
\url{http://nase-rec.ujc.cas.cz/archiv.php?art=5348}.


% TODO!!!
\bibitem{prizvuk}
 {\sl Marta Šimečková: Jak je to se slovním přízvukem v~češtině.} (How is
 it with a word-accent in Czech)
 \url{http://www.vaseliteratura.cz/teorie-literatury/144-slovni-prizvuk}


\bibitem{rymy}
 {\sl Pavel Šrubař: www.rymy.cz} (Rhymes.cz)
 \url{http://www.rymy.cz/rymy.htm}




\bibitem{zeman}
 {\sl Jiří Zeman: Ještě k přízvukování prvotních jednoslabičných
 předložek}
(Once again about a stress of monosylabic prepositions). [1983] Published in {\sl Naše řeč}
(Our language),
{\sl year 66, number 4, pg. 192-197}. Available at
\url{http://nase-rec.ujc.cas.cz/archiv.php?art=6402}.



\bibitem{jurafsky}
 {\sl  Dan Jurafsky and James H. Martin:	
 Speech and Language Processing
 }




\end{thebibliography}



%%%%%%%%%%%%%%%%%%%%%%%%%%%%%%%%%%%%%%%%%%%%



\end{document}

























%\centerline{
%Late {\bf eve}ning, {\bf on} the {\bf first} of {\bf May}—
%}
%
%\centerline{
%{\bf The} twilit {\bf May}—the {\bf time} of {\bf love}.
%}
%\centerline{
%{\bf Mel}tingly {\bf called} the {\bf turt}le-{\bf dove},
%}
%\centerline{
%Where {\bf rich} and {\bf sweet} {\bf pi{\bf ne}woods {\bf lay}.
%}
