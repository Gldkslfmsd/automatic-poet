\documentclass[a4]{article}
%\usepackage{czech}
\usepackage{graphicx}
\usepackage[utf8]{inputenc}   % pro unicode UTF-8
\usepackage{booktabs}
\usepackage{hyperref}
%#\usepackage{qtree}
%\usepackage{graphviz}
\usepackage{authblk}
%\usepackage{tikz-dependency}
\usepackage[ampersand]{easylist}

\begin{document}

\title{Automatic accentual-syllabic poet}

\author{Dominik Macháček}
\affil{
Saarland University, Saarbrücken, Germany,
dominik.machacek@matfyz.cz
}

\date{\today}

%\pacs{PACS numbers go here. These are classification codes for your  research. See {\tt http://publish.aps.org/PACS/} for more info.}
\maketitle

\begin{abstract}
TODO: An abstract is a great convenience for the reader and is required by all journals.
\end{abstract}


\section{Introduction}

Our task is to create a generator of an accentual-syllabic poetry.

Verses in this kind of poetry are restricted by the rhythm of accents, all
verses must have given number of syllables and the verses in a strophe
rhyme by the chosen pattern. See example in figure \ref{may}.

\begin{figure}[ht]
\label{may}
\centerline{
I {\bf know} a {\bf glade}, spring {\bf crys}tal {\bf clear},
}
\centerline{
in {\bf dee}pest {\bf wood}land, {\bf crowned}
}
\centerline{
by {\bf sha}dy {\bf ferns} in {\bf si}lhou{\bf ettes},
}
\centerline{
red {\bf hea}ther {\bf all} a{\bf round}.\footnotemark
}
\caption{Bold syllables are stressed, other syllables are non-stressed.
There is a regular pattern of stressed and non-stressed
syllables in each verse. Second and fourth verse rhyme.
}
\end{figure}


\footnotetext{From "The Crystal Spring" by J. V. Sládek, translated by
Václav Z J Pinkava.
Available at \url{http://www.vzjp.cz/basne.htm\#Sladek}.
}


% Summary?


In this project we make an automatic generator of accentual-syllabic
poems. Its input are a poetic form, namely a pattern of stressed and
non-stressed syllables in each verse, and a raw coherent Czech text. The
whole process of generating is fully automatic, no manuall anotation of
the text is needed.

On the output there are poems compounded from newly-created nonsense words
reminding Czech language. They are pronouncable and they precisely
fit into the given poetic form.

\section{Motivation}

Accentual-syllabic poetry is very popular genre in many national cultures
including Czech, English and German. Poems written by human
poets are usually written in some variety of natural language, their words
have meaning and the whole poem has some intention.
To write a good poem is a difficult task, a poet must choose proper words
and fit them into a pleasant form. 

Our motivation is the fact that such poems will be unique, interesting and
entertaining pieces of art. They could also be published for
general public and draw attention to the whole field of Computational Linguistics.

\section{Related works}

TODO

\section{Solution}

Generator's workflow consists of two phases, training and generating. 

In
training phase, an input text is preprocessed and splitted to syllables.
Then stresses are indicated.  Finally, a language model is created from list
of syllables and their accents.

In generating phase, a language model is used for creating of a poem by given form.

More detailed descriptions of each mentioned process follow in special subsections.

Our generator can work for arbitrary natural language, but we work with
Czech. Reasons for this decision is that there exist many
resources of Czech texts, we can use an automatic tool for splitting text to syllables,
and the word-stress in Czech language is very regular so we can indicate
it automatically. Czech is also a native language of the author of this
project, so we are able to evaluate the quality of resulting poems. 

For some other languages we should use either a special tools for splitting
to syllables and for accent indication (which we don't have for any other
language including English),
or we could use a corpus annotated with syllables and accents, but this
restricts the domain and size of input texts.


\subsection{Text preprocessing}

In text preprocessing we do tokenization and sentence segmentation, because
syllabification and accentification requires single sentences. We also
remove punctuation and transform text to lowercase. This decreases number of
unique syllables and makes language model more inovative.

For tokenization and segmentation we used NLTK. % reference

\subsection{Syllabification}

For automatic syllabification we used "sekáček". % TODO
It's a Python implementation of rules created by experts on this issue, who
published it in % reference

\subsection{Accents}

Accents in Czech 

\subsection{Language model}

\subsection{Generating}

\section{Example outputs}

\section{Conclusion}


\begin{thebibliography}{99}

\bibitem{lamport}
 {\sl LaTeX : A Documentation Preparation System User's Guide and Reference Manual}, Leslie Lamport [1994] (ISBN: 0-201-52983-1) pages: xvi+272.

\bibitem{latt}
I.M. Smart {\it et al.}, J. Plumb Phys. {\bf 50}, 393 (1983).

\end{thebibliography}



%%%%%%%%%%%%%%%%%%%%%%%%%%%%%%%%%%%%%%%%%%%%



\end{document}

























%\centerline{
%Late {\bf eve}ning, {\bf on} the {\bf first} of {\bf May}—
%}
%
%\centerline{
%{\bf The} twilit {\bf May}—the {\bf time} of {\bf love}.
%}
%\centerline{
%{\bf Mel}tingly {\bf called} the {\bf turt}le-{\bf dove},
%}
%\centerline{
%Where {\bf rich} and {\bf sweet} {\bf pi{\bf ne}woods {\bf lay}.
%}
